\documentclass{article} % For LaTeX2e
\usepackage{nips15submit_e,times}
\usepackage{hyperref}
\usepackage{url}
\usepackage{times}
\usepackage{url}
\usepackage{graphicx}
\usepackage{amsmath}
\usepackage{bbm}
\usepackage{hyperref}
\usepackage{scabby}
\usepackage{tikz}
%\documentstyle[nips14submit_09,times,art10]{article} % For LaTeX 2.09

%\newcommand\ac[1]{}
%\newcommand\pl[1]{}
%\newcommand\pldone[2]{}
\newcommand\ac[1]{\textcolor{red}{[AC: #1]}}
\newcommand\pl[1]{\textcolor{red}{[PL: #1]}}
\newcommand\pldone[2]{\textcolor{green}{[PL: #1]}\textcolor{green}{[AC: #2]}}

\providecommand{\byt}{\hat{\by}}
\providecommand{\bys}{{\by^*}}
\providecommand{\yt}{\hat{y}}
\providecommand{\ys}{{y^*}}
\providecommand{\Regret}{\operatorname{Regret}}
\providecommand{\p}{p_{\theta}}
\providecommand{\e}{\epsilon}

\providecommand{\yh}{\tilde{y}}
\providecommand{\ofi}[1]{^[#1]}

\providecommand{\request}{\operatorname{\textsc{request}}}
\providecommand{\wait}{\operatorname{\textsc{wait}}}
\providecommand{\turnin}{\operatorname{\textsc{turn-in}}}

%\input std-macros.tex


\title{Fake it until you make it:\\Asynchronous Bayesian Active Classification\\with Crowds}

\author{%
Keenon Werling\\
Department of Computer Science\\
Stanford University\\
\texttt{keenon@cs.stanford.edu} \\
\And
Arun Chaganty\\
Department of Computer Science\\
Stanford University\\
\texttt{chaganty@cs.stanford.edu} \\
\And
Percy Liang\\
Department of Computer Science\\
Stanford University\\
\texttt{liang@cs.stanford.edu} \\
\And
Chris Manning\\
Department of Computer Science\\
Stanford University\\
\texttt{manning@cs.stanford.edu} \\
}

% The \author macro works with any number of authors. There are two commands
% used to separate the names and addresses of multiple authors: \And and \AND.
%
% Using \And between authors leaves it to \LaTeX{} to determine where to break
% the lines. Using \AND forces a linebreak at that point. So, if \LaTeX{}
% puts 3 of 4 authors names on the first line, and the last on the second
% line, try using \AND instead of \And before the third author name.

\newcommand{\fix}{\marginpar{FIX}}
\newcommand{\new}{\marginpar{NEW}}

%\nipsfinalcopy % Uncomment for camera-ready version

\begin{document}

\usetikzlibrary{positioning}

\maketitle

\pl{I'd prefer a shorter title;
  not sure 'Fake it until you make it' is the right metaphor anyway
  since 'faking it' here actually means doing a good job too;
  if you want something fun, something along 'training wheels' or 'cyborg'?
  Something about smoothly transitioning from humans to machines in an online manner?
  Actually, why not call it LENSE?
}

\pl{not sure we actually want to pivot around the word 'active classification' so much;
I think we're sufficiently different from it; not only are they typically
not querying humans, but there's not a sense that the learning algorithm
gets better and the knowledge gets transfered from humans to computers,
which is \emph{the key point}}

\begin{abstract}
Our goal is to deploy a high-accuracy system starting with zero training examples.
We consider an “on-the-job” setting, where as inputs arrive, we use real-time crowdsourcing to resolve uncertainty where needed and output our prediction when confident. As the model improves over time, the reliance on crowdsourcing queries
decreases. We cast our setting as a stochastic game based on Bayesian decision
theory, which allows us to balance latency, cost, and accuracy objectives in a principled way. Computing the optimal policy is intractable, so we develop an approx-
imation based on Monte Carlo Tree Search. We tested our approach on three
datasets---named-entity recognition, sentiment classification, and image classification. On the NER task we obtained more than an order of magnitude reduction in cost compared to full
human annotation, while boosting performance relative to the expert provided labels. We also achieve a $8\%$ \fone{} improvement over having a single
human label the whole set, and a $28\%$ \fone{} improvement over online learning.
\end{abstract}

% V1
% Recent work in crowd-powered products has demonstrated the potential of using ``crowd workers'' to power live, low latency systems that accomplish AI complete tasks. 
% These systems suffer from high scaling cost, and the unreliability of workers.
% We propose an \textit{active classification} approach to dramatically reduce the scaling cost and improve the accuracy of such systems.
% In order to also achieve low latency, we investigate \textit{asynchronous} active classification, where multiple requests can be simultaneously ``in-flight.''
% This leads to the twin challenges of how to behave optimally in the presence of asynchronous noisy oracle queries that have not yet returned, and when to return a classification that is ``good enough'' in such a setting.
% We first reduce the problem of optimal asynchronous active classification to a Partial Monitoring game, by making use of Bayesian decision theory.
% We show a bound on achievable regret in this setting, and demonstrate practical heuristics that approach this bound.
% We also show adaptations for traditional Active Learning algorithms to our setting.
% We show empirical demonstration of our proposed algorithms, which demonstrate dramatic improvement over the human-only, machine-only, and baseline human-machine hybrid alternatives in the cost-delay-accuracy tradeoff surface.

% V2?
% Recent work in real-time crowd-powered products has demonstrated the possibility of using human ``crowd workers'' to power live, low latency products that accomplish AI complete tasks. Human-only solutions remain expensive, and do not scale into production. We propose a \textit{active classification} approach to dramatically reduce the scaling cost of such systems, backed by a pool of unreliable crowd-workers. In order to achieve this, we investigate \textit{asynchronous active classification}, where multiple requests can be simultaneously ``in-flight.'' This paper analyzes the \textit{asynchronous requests problem} of how to behave optimally in the presence of asynchronous oracle queries that have not yet returned, and the \textit{optimal stopping problem} of when a classification is ``good enough'' in such a setting. Our solutions to these problems enable a system that shows dramatic improvement over the human-only, machine-only, and baseline human-machine hybrid alternatives in the cost-delay-accuracy tradeoff surface.
% With sufficient labeled data and research effort, supervised learning has been applied in performance critical domains.
% Yet, often logistical/financial constraints mean that data can only be collected and labeled in the presence of a working system, or that insufficient data can be collected to achieve performance requirements.

% V3
% We propose a crowd/machine learning hybrid approach to apply supervised learning in performance critical domains where ``cold start'' barriers exist: we allow our model to query the crowd in real-time {\em at test time}, and learn from the crowd responses online to improve performance and reduce costs on future inputs.
% %Querying humans in real time introduces latency, noise, and cost into the system, which must be minimized.
% Querying humans introduces latency, noise, and cost into the system, which must be minimized.
% We explicitly manage these three key objectives by casting the problem as a Bayesian decision theoretic control problem, and drawing on techniques from game-playing literature for real-time solutions.
% The resulting system provides high quality responses in real-time while starting {\em with no training data\/}: we have achieve $> 90\%$ F1 on the three tasks studied at a low cost, despite noisy crowd worker performance.
% % outperform baselines?

% V?
%Learning with little or no data is an important but understudied problem.
%For most tasks, the current paradigm of first collecting a training dataset and then training a classifier requires a substantial amount of data to be labeled to train a very accurate classifier.
%We propose a  that redistributes the same human effort to achieve
%better results faster: 
%query humans in real time at test time to train a machine learning system {\em
%on the job}.
%%\ac{is it clear what we're doing?}
%%The system only queries humans when it is unsure of its output and learns to 
%Querying humans introduces cost and latency but can improve accuracy over our
%model's prediction.
%We explicitly manage these three key objectives by casting the problem as a
%Bayesian decision theoretic control problem
%and draw on techniques from the game-playing literature.
%The resulting live system provides high quality responses in real-time while
%starting {\em with no training data\/}: we maintain $> 90\%$ F1 consistently on
%a stream of test inputs on the three tasks studied at low cost and latency,
%despite noisy labels from crowd workers.
% outperform baselines?

% V4
%Our goal is to deploy a high-accuracy, real-time classification system, without expensive up-front data labelling.
%We describe an ``on-the-job'' classification system, where as queries arrive, we use real-time crowdsourcing to resolve uncertainty where needed and output our prediction when confident.
%As the model improves over time, the reliance on crowdsourcing queries decreases. 
%We cast our setting as a stochastic game based on Bayesian decision theory, which allows us to balance latency, cost, and accuracy objectives in a principled way.
%Computing the optimal policy is intractable, so we develop an approximation based on Monte Carlo Tree Search.
%We tested our approach across three tasks---named-entity recognition, sentiment classification,
%and image classification.
%On the named-entity recognition task, we obtained a 6-7 fold reduction in cost compared to human classification, which requires multiple human votes to achieve the same level of accuracy.
%We also achieve a 17\% F1 improvement over taking a single human vote as the answer to each query, and a 28\% F1 improvement over an online learning classifier that cannot query humans.


%\section{TODO}

\subsection{Arun}

\begin{itemize}
  \item I don't like that "money" and "time" are explicit variables in the loss function, mainly because it seems too concrete for abstract math. We can choose an specific loss function at a later time dependent on those parameters. I might yet come around to this notation.
\end{itemize}


\section{Introduction}

\begin{epigraph}
``It ain't what you don't know that gets you into trouble.\\
It's what you know for sure that just ain't so.'' \\
-- Mark Twain
\end{epigraph}

For the practitioner, supervised learning is largely a solved problem and software to efficiently train classifiers in a variety of domains is readily available.
Yet, adoption remains limited because the deployment of a learning system requires extremely accurate classifiers: mistakes cost business.
The typical solution to this problem is to use more labeled data, which recent crowdsourcing platforms such as Amazon Mechanical Turk or CrowdFlower have made relatively affordable to obtain\findcite{citefest!}.
However, this not only presupposes the ready availability of large amounts of unlabeled data, but also advocates a long, expensive data collection process for uncertain improvements in accuracy.
% This two-stage process fundamentally limits the accuracy we can obtain.
An alternate approach is to use cut out the middle man altogether and use the crowd to label all examples in real-time\cite{cheng2015flock}. 
While this allows us to ensure accurate responses, it becomes prohibitively expensive to scale to more data. 

Our approach interpolates between these two regimes: we query the crowd in real-time when our model is unsure and learn on the crowd responses to improve on future input.
This results in a system that provides users high quality responses in real-time while starting {\em with no training data}.
In this paper, we explicitly address three main challenges that arise in doing so: keeping costs low, guaranteeing low latency and maximizing accuracy.

% Cost and Low-latency
We focus on structured classification problems using conditional exponential families, a general model class that has been used to \todo{citefest}. In such a model, we are given an input, $\bx$, and must predict a number of labels $\by = y_1, \ldots y_n$.

We treat crowd workers as a resource that can provide noisy measurements of some subset of the labels.
Under time and budget constraints, we must optimize over which labels to query when. 
Often, several queries are required on a particular label because of annotator errors by the crowdworker, while at other times it is better to distribute queries across labels.
Similarly, observing a label from one worker lets us decide better which label to query next, but we might not have to time to wait for the response.

We propose an active classification approach using Bayesian decision theory that is able to make these complex behavioral decisions.
This quickly leads to an intractable optimization problem that grows exponentially in complexity with the number of label queries we might ask on a single example.
We propose a novel approximation based on Monte Carlo tree search that retains the behavior of the original Bayesian approach while being computationally tractable.

\todo{(arun): We should make the distinction with active learning much stronger since that's what everyone thinks we're doing. I've moved the related work section to the latter parts because I feel like it's easier to compare our work with existing work given our model.} 

Finally, in practice, labels from crowd workers are often inaccurate.
We use the measurements framework of Liang et.\ al\cite{liang09measurements} to incorporate noisy responses from crowdworkers in our model.
Having an accurate estimate of the error rates for crowd workers is essential to accurately predicting how many queries are required.
Prior work\findcite{unsupervised crowd labeling} uses inter-annotator agreement to predict per-user error rates in an unsupervised manner. 
The online nature of our task limits the number of responses we have on the same label.
Instead, we learn the error rates in an unsupervised fashion using online EM, which also allows us to incorporate unlabeled data.

% Experiments
We evaluate our system on four different tasks: named entity recognition, information extraction from user generated content, image classification and sentiment classification on tweets.
We show that by querying crowdworkers at classification time, we can significantly outperform a system trained on fully labeled data, for a fraction of the cost of the baseline of asking crowdworkers for labels for each example, as well as the system trained on fully labeled past data.
On X of the Y tasks, we produce a classifier comparable with the state of the art while obtaining a much smaller subset of the training labels noisily from the crowd. 
In fact, we are comparable with state of the art-ish with a handful of the training labels.
An open-source implementation of our system will be made available.

% Recent work has shown that it is possible to use real-time on-demand workers to power everything from AI-complete email clients~\cite{kokkalis2013emailvalet} to real-time activity surveillance and classification~\cite{lasecki2013real}.
% These purely crowd-based solutions are prohibitively expensive at scale.
% Powering the crowd-based email client \textit{EmailValet}~\cite{kokkalis2013emailvalet} for a single end user for a year costs over \$400.
% 
% These systems typically work by ``pooling'' on-demand workers from high latency job-posting platforms like Amazon Mechanical Turk or CrowdFlower on a website designed by the system architect~\cite{lasecki2011real}.
%  The ``pooling'' process can take several minutes, but once in place the workers can be queried at very low latencies by pushing requests to their web-browsers.
%  This pool of workers can demonstrate high rates of turnover, and unreliability amongst individual annotators.
% 
% Existing systems query this pool directly, allowing for annotator noise by incorporating consensus building systems like voting and chat.
% 
% 
% Active classification~\cite{greiner2002learning}, a close sibling of active learning, is a setting in which a classifier is allowed to query for more information, at some cost, before turning in classifications.
%  This active classifier is intended to reduce its need for costly, slow human labels over time by learning from past observations.
%  We propose to adapt the active classification framework to the pooled-worker setting to query this pool more cheaply, accurately, and quickly, without sacrificing the advantages live of crowd-powered interfaces.
% 
% Previous work in online active learning (which is closely related to what we're proposing) has focused on multi-class classification (\cite{chu2011unbiased},~\cite{agarwal2013selective},~\cite{cheng2013feedback},~\cite{vzliobaite2011active},~\cite{helmbold1997some}).
%  Multi-class classification is an insufficiently rich primitive to handle many of the tasks that crowd-workers are enabling in existing systems, like information extraction or object detection.
%  Instead, we will build our platform around arbitrary log-linear markov network classification, where we assume it is possible to query workers for opinions on individual nodes.
%  Thus each ``active classification'' in our proposed setting is instantiate with a markov network and involves using model priors trained on previously seen data to choose to query the worker-pool for opinions about nodes, and then returning a classification informed by those opinions.
% 
% 
% This setting poses several distinct challenges that have not been sufficiently addressed in previous literature.
%  We need to be sensitive to time delays, returning results at least as quickly as the pure-crowd baseline we intend to improve upon.
%  We also need to be sensitive to inaccurate oracles.
%  These two criteria, in the pooled-worker setting, means that we need an active classifier who is able to hide latencies of redundant queries by launching them \textit{asynchronously}.
%  This leads to the two challenges we will address in this paper, which can both be clustered under \textit{optimal asynchronous behavior}: we need to be able to handle the decision to ask for another query or turn in existing results in the presence of ``in-flight requests,'' which can fail due to worker turnover, where our loss term is sensitive to time delay.
%  We draw inspiration from work in Bayesian Active Learning to tackle these problems.




\section{Problem setup}
\label{sec:model}

\begin{figure}[t]
  \begin{centering}
  \includegraphics[width=1.0\textwidth]{figures/intro-banner.pdf}
  \end{centering}
  \caption{
    Named entity recognition on tweets in on-the-job training.
    %(1) The system first runs a
%pre-trained model, discovers that the token ``George'' is ambiguous, and then
%asks a human for a label. The human label is returned in a few seconds,
%incorporated into the model, and the model then decides it has enough
%information to turn in a classification.
}
\label{fig:crf}
\end{figure}

\afterpage{\noindent\begin{figure}
  \begin{centering}
    \begin{subfigure}[b]{0.58\textwidth}
      \includegraphics[width=\textwidth]{figures/piano-roll.pdf}
      \caption{Structured predicition behavior over time}
    \end{subfigure}
    \hfill
    \begin{subfigure}[b]{0.38\textwidth}
      \includegraphics[width=\textwidth,height=0.23\textheight,keepaspectratio]{figures/single-move.pdf}\\[1.7ex]
      \caption{Single-query gametree}
    \end{subfigure}
  \end{centering}
\label{fig:piano-roll}
\caption{Example behavior while running structure prediction on the tweet ``Soup on George str.''}
\end{figure}}

% Define structured prediction
Consider a structured prediction problem from input $\bx = (x_1, \dots, x_n)$ to output $\by = (y_1, \dots, y_n)$.
For example, for named-entity recognition on tweets,
$\bx$ is a sequence of words in the tweet (e.g., \nl{on George str.})
and $\by$ is the corresponding sequence of tags (e.g., \scnone{} \scloc{} \scloc{}).
The full set of tags of \scper{}, \scloc{}, \scres{}, and \scnone{}.

% Basic setting
In the \emph{on-the-job training} setting, inputs arrive in a stream.
On each input $\bx$,
we make zero or more queries $q_1, q_2, \dots$ on the crowd to obtain tags
(potentially more than once)
for any positions in $\bx$.
The responses $r_1, r_2, \dots$ come back asynchronously,
which are incorporated into our current prediction model $p_\theta$.
\figureref{piano-roll}(left) shows one possible outcome:
we first query position $q_1 = 2$ (\nl{George}), which later returns $r_1 = \scloc$;
in the meantime, we have queried $q_2=3$ (\nl{str.}) and gotten back $r_2 = \scloc$.
When we have sufficient confidence about the entire output,
we return the most likely prediction $\hat \by$ under the model.
Each query $q_i$ is issued at time $s_i$ and the response comes back at time $t_i$.
Assume that each query costs $m$ cents.
Our goal is to choose queries to maximize accuracy, minimize latency and cost.

% More intuition / connections
We make several remarks about this setting:
First, we must make a prediction $\hat \by$ on each input $\bx$ in the stream,
unlike in active learning, where we are only interested in the pool or stream of examples
for the purposes of building a good model.
% PL: though active learning will query hard examples too
Second, we evaluate on accuracy $\accuracy(\by, \byt)$ against the true tag sequence $\by$
(on named-entity recognition, this is the F$_1$ metric),
but $\by$ is never actually observed---the only feedback is via the responses,
like in partial monitoring games \citep{cesabianchi06regret}.
Therefore, we must make enough queries to garner sufficient confidence
(something we can't do in partial monitoring games)
on each example from the beginning.
Finally, the responses are used to update the prediction model, like in online learning.
This allows the number of queries needed (and thus cost and latency) to decrease over time
without compromising accuracy.

%We receive a stream of inputs $\bx\oft1, \bx\oft2, \ldots, \bx\oft N$ with corresponding {\em unobserved\/} true output $\by\oft1, \by\oft2, \ldots, \by\oft N$ that we would like to predict.

% PL: too much detail about strategy
% at this point, just the get dynamics of the environment set up.
%Let us make our problem setting concrete with an example, depicted in \figureref{piano-roll}.
%We receive an input ``Soup on George str \#Katrina'' ($\bx$) and must produce
%an output ($\byt$) that labels words with the tags.
%Initially, our model is uncertain about both ``George'' and ``str''.
%We first query the crowd for a label on ``George'' ($q_1$) and then for a label on ``str'' ($q_2$). 
%After some time, the crowd responds with the label \scloc{} on both ``George'' ($r_1$) and ``str'' ($r_2$).
%At this point, the model is confident in its labeling, and we can turn in its
%prediction $\byt$.
%By querying the crowd queries $\bq = (q_1, q_2)$, we are not only able to
%produce a more accurate output, but can also use the responses $\br = (r_1,
%r_2)$ to train a better model.

%Of course, querying the crowd also incurs a cost for the queries as well as a
%time delay, $t$: $C(\bq, t)$ \pl{give example numbers that we used}

%In general, we want to maximize our accuracy by making queries while trading
%off the cost and time delay introduced.

% We cast this problem in the Bayesian decision theoretic framework: our objective is to maximize our expected utility under our current model,
% $\p(\by \given \bx, \br)$:
% \begin{align*}
%   u &= \E_{\by \sim \p(\cdot \given \bx, \br)}[1 - \ell(\by, \byt) + C(\bq, t)].
% \end{align*}

%\ac{Probably should talk more about training.} % PL: not really

%We receive a stream of inputs $\bx\oft1, \bx\oft2, \ldots, \bx\oft N$ with corresponding {\em unobserved\/} true output $\by\oft1, \by\oft2, \ldots, \by\oft N$ that we would like to predict.
%For each input $\bx$, we may query the crowd several times. Let $Q = \{q_1, \ldots, q_m\}$ be the set of queries.
%Finally, we observe responses $R = \{r_1, \ldots r_m\}$ to our queries after some time $t$.
%Using the information from these queries, our model makes the prediction, $\byt\oft{t}$.
%Our goal is to maximize the accuracy, trading off cost and latency as specified by a given objective function.
%
%More formally, suppose the output $\by\oft{t}$ has $n$ parts: $\by\oft{t} = y\oft{t}_1, \dots, y\oft{t}_n$.
%Let $q\oft{t} \in \{1, \ldots, n\}$ be a request for the label $y\oft{t}_q$,
%let $Q\oft{t} = (q\oft{t}_1, \dots, q\oft{t}_m)$ be a sequence of queries made on the $t$-th input and
%let $\tau\oft{t}$ be the time taken to make the prediction $\byt\oft{t}$. 
%We would like to minimize the cumulative loss, 
%following objective:
%\begin{align}
%  \sL &= \sum_{t=1}^T \ell_{\rmclass}(\by\oft{t}, \byt\oft{t}) + C(Q\oft{t}, \tau\oft{t}), \label{eqn:objective}
%\end{align}
%where $\ell_{\rmclass}$ is a given misclassification loss function, e.g.\ the Hamming loss, and $C$ is a given cost function.
%\ac{Concern: by using the cumulative loss, is there an expectation that we will ``model'' input and optimize for the future?}
%
%We now describe our choice of models for prediction, human error and latencies and return to optimizing \equationref{objective} in \sectionref{async}.
%

%\section{Theoretical Bounds on Active Classification Performance}
\label{sec:bounds}

We're interested in providing extreme limits of classifier performance under an unrealistic set of assumptions, as a way to gain intuition about the setting and the limitations of our approach.
It's trivially easy to see that it is possible to spend no money, and no time, during classifications. Simply return null on every request.
It's more interesting to evaluate if it's possible to return a perfect classification every time, and how much we have to give up in average cost to achieve perfection.
Let's assume that our temporal and financial budgets are infinite, we have an infinite number of humans on call, and we have an accurate model of the confusion matrix of humans $C$.
If all of these conditions are met, are there any guarantees we can make about our classifier?
Trivially, if our classifier always returns a uniform distribution over all possible output labels, and we query an infinite number of humans, it is possible to achieve arbitrary accuracy.
More interestingly, are there non-trivial bounds we can set on the cost-guarantee tradeoffs?

As an overview, our analysis will first define the value of valid priors on the data in terms of number of human queries required, under some simplifying assumptions about human error.
We will then focus on the challenge of not being able to absolutely trust model confidence estimates.
We define a rigorous way of conceptualizing ``trust'' in our confidence values, which requires defining the set of ``true confidence'' estimates allowable by the data for a given input.
Then we define bounds of expected classification performance given a choice for this ``trust,'' and demonstrate that more trust leads to higher error rates and lower costs.

\subsection{True Confidence Values}

First we must define the set of ``true confidence'' distributions $\Delta$.
Let data arrive in the form of $(\bx_i,\by_i)$ pairs, where $\by_i$ is the always observed set of variables we condition on, and $\bx_i$ is the true observed outcome.
Given a set $\Gamma$ of IID pairs, a norm $N$, and a feature mapping $f(\by) = \bz \in \sR^m$,
 let the ``$\epsilon$-neighborhood'' of $\by_t$ (notation $\Gamma(\epsilon, \by_t)$) be defined as all pairs in $\Gamma$ within the norm ball under $N$ of size $\epsilon$:
\[\Gamma(\epsilon, \by_t) = \{\bx_i \in \Gamma | ||f(\by_i) - f(\by_t)||_N \leq \epsilon\}\]
Every $\epsilon$-neighborhood, for $\epsilon \in \sR^+$, defines a natural frequentist estimate for $P(\bx_t | \by_t)$:
\[P(\bx_t | \by_t; \epsilon) \approx \sum \frac{\Gamma(\epsilon, \by_t)}{|\Gamma(\epsilon, \by_t)|}\]
Intuitively, $\epsilon$ has two limit conditions:
\[\lim_{\epsilon\to\infty} P(\bx | \by; \epsilon) = \sum \frac{\bx_i}{|\Gamma|}\]
\[\lim_{\epsilon\to0} P(\bx | \by; \epsilon) = \sum \frac{\bx_i | \by_i = \by_t}{\sum \mathbbm{1}\{\by_i = \by_t\}}\]
Let the set of ``true confidence'' distributions $\Delta$ be defined as
\[\Delta = \{P(\bx | \by; \epsilon) | \epsilon \in \sR^+\}\]

\subsection{The Value of Valid Priors}

Assume we are given a vector $\by$, and we want to predict $\bx = \argmax_{\bx} P(\bx | \by)$, and we have access to an infinite number of human labelers, as explained above.
We are tasked with, regardless of cost, ensuring that when given an infinite number of $\by_i$, we achieve an error rate of $e_{\text{acceptable}}$
\[\lim_{i\to\infty} \frac{\sum_i \frac{||\bx_{i,\text{true}} - \bx_{i,\text{guess}}||_{\infty}}{|\bx|}}{i} \leq e_{\text{acceptable}}\]
In practice, this implies the policy of turning in a vector of guesses $x$ only when
\[\min_{i} \max_{k} P(\bx_i = k | \by) > 1 - e_{\text{acceptable}}\]
Until that condition is met, we must continue asking queries of humans.
For simplicity of analysis, to derive a lower bound on value of information, let's assume that we're confined to treating each variable in $\bx$ independently.
As we arrive at a variable $i$, any previous human queries asked of nodes $j < i$ are already incorporated in our observed marginals for $\bx_i$.

Let's assume that our model delivers perfectly accurate marginals $m_i$ for the classes $\bx_i$.
Formally, all marginals $m_i \in \Delta$.
How does a setting of $m_i$ effect expected number of queries to achieve $\min_{i} \max_{k} P(\bx_i = k | \by) > 1 - e_{\text{acceptable}}$?

Let's assume a simplified model of human error for this analysis, where humans are defined as follows:
\begin{equation}
    Q(o_i, X_j) \sim
    \begin{cases}
       \text{uniformly drawn from } D_j = \{1 \ldots K_j\}, & \text{with probability}\ \epsilon \\
      h_{\text{true}}(X_j), & \text{otherwise}
    \end{cases}
\end{equation}

Then we can easily derive that a human query is worth a \todo{math} reduction in expected $E[\min_{i} \max_{k} 1-P(\bx_i = k | \by)]$.
This then leads to the recurrence for expected number of queries remaining given $c$.

\[E[\text{\# queries}] = \text{\todo{recurrance}}\]

\subsection{The Worst-case Cost of Invalid Priors}
\label{sec:worst-case}

Let's assume that our model's prior is not valid ($m_i \not\in \Delta$), but is ``close,'' as in $\min (||M(\by_i, j) - \delta||_N | \delta \in \Delta) < \gamma$, for some $\gamma \in \sR^+$, and norm $N$.
What's the worst that could happen to our accuracy, assuming we're still running the algorithm in the last section (turn in when $\min_{i} \max_{k} P(\bx_i = k | \by) > 1 - e_{\text{acceptable}}$, otherwise query).
We have $m_i \not\in \Delta$ and $m_{i,\text{true}} \in \Delta$ where $m_{i,\text{true}} = argmin_{\delta} (||m_i - \delta||_N, \delta \in \Delta)$.

Given that our humans are drawn from \todo{$H = $humans cross $m_{i,\text{true}}$}, we have that our algorithm will terminate when
\[\max_{k} (m_i \times H^{\text{\# queries}})_k > 1 - e_{\text{acceptable}}\]
This will produce more errors than $e_{\text{acceptable}}$, as described by:
\[g(\gamma) = \text{\todo{math}}\]

\subsection{Trust Guarantees for Model Confidence Estimates}

Given a model $M$ which is capable of providing confidence estimates, we would like to quantify how much trust we can place in these estimates.
Let $M(\by_i, j) = c_j \in R^k$, where $c_j$ is the vector of marginal class probabilities for node $j$ predicted by the model.
Let $\gamma \in R^+$ be constants $\beta \in R^+$, and $N$ be a norm, then we define a trust guarantee for our model to be a statement of $(\gamma, \beta, N)$ where
\[P(\min (||M(\by_i, j) - \delta||_N | \delta \in \Delta) > \gamma) \leq \beta\]
A set of approximate trust guarantees $(\gamma, \beta, N)$ can be estimated in practical contexts by observing $\Delta$ for a finite set of $\epsilon$ on a held out dataset, and calculating directly.
Note that it is possible to produce guarantees that set $\beta = 0$, by simply setting $\gamma = \infty$.
Likewise $\gamma = 0$ is easily achieved with $\beta = \infty$.
These are not useful bounds.

\subsection{Model Under-Confidence}

In practice, $\Delta$ is never observed, so we have to perform some approximations if we want to provide real guarantees.
We know, however, a good estimate of one member of $\Delta$: the underlying class distribution $\delta_{\text{class}} \in \Delta$.
It's intuitive that model underconfidence, in the extreme case always predicting $\delta_{\text{class}}$, yields both $\gamma \approx 0$ and $\beta \approx 0$.
By Chebyshev, assuming some underlying data variance $\sigma$, we have the following guarantees for our massively underconfidence model:
\[\gamma = k\sigma, \beta = \frac{1}{k^2}, \text{for all} k \in \sR\]
In this way we can achieve a tight guarantee, but the information provided by our model is worth very little.

\subsection{$\gamma$-$\beta$ Query-Bound Tradeoffs}

Given a choice of a $(\gamma, \beta, N)$ trust guarantee, how can we describe model performance?

For argument, assume that for now our $\beta = 0$ and $\gamma < \infty$.
Then we can say with certainty that all our $M(\by_i, j) = c_j$ are within $\gamma$ of some true $\delta \in \Delta$.
Assume our target is some $e_{\text{acceptable}}$ error rate, then we can use the formula from \ref{sec:worst-case} to solve for a new $e_{\text{acceptable $\gamma$-adjusted}}$ given $e_{\text{acceptable}}$, given our $\gamma$ guarantee:
\[\text{todo{math}}\]
This results in an expected number of queries, as a function of $e_{\text{acceptable}}$ and $\gamma$.
\[\text{todo{math}}\]

Now consider $\beta > 0$. In the worst case all the instances where our model prediction is worse than $\gamma$, $\min (||M(\by_i, j) - \delta||_N | \delta \in \Delta) > \gamma$, we just get wrong.
That means that our new expected error rate is $e_{\text{acceptable $\gamma$-adjusted}} + \beta$.
We do not attempt any guarantees that get below $\beta$ in expected error rate.

This means that for larger $\gamma$, we can provide an arbitrary expected error rate, at much greater cost in queries, since we know how to compensate correctly for a $\gamma$ error rate.
Larger $\beta$ will allow smaller $\gamma$, and consequently a smaller number of queries, but the floor on our minimum expected error rate is increased.

\section{Choosing when and what to query}
\label{sec:async}

% Restate our goal
As described in \sectionref{model},
we would like to develop an algorithm that can select and schedule several requests for labels, $R\oft t = \{r_1\oft t, \ldots, r_m\oft t\}$, to maximize our accuracy on predictions, $\ell(\by\oft t, \byt\oft t)$, while trading off the cost and response time, $\tau\oft t$. 
\todo{connective}
\begin{align*}
  \sL &= \sum_{t=1}^T \ell(\by\oft{t}, \byt\oft{t}) + C(R\oft{t}, \tau\oft{t}),
\end{align*}
where $C$ is a given cost function.

\begin{figure}
  \includegraphics[width=0.49\textwidth,height=0.23\textheight,keepaspectratio]{figures/mcts.pdf}
  \hfill
  \includegraphics[width=0.49\textwidth,height=0.23\textheight,keepaspectratio]{figures/mcts_simple.pdf}
  \caption{Monte Carlo tree search}
\label{fig:mcts}
\end{figure}

We cast the problem in the Bayesian decision theoretic framework using an expectimax game tree.
The game alternates between our agent (max nodes) and the environment (expectation nodes).
The state keeps track of time, responses received so far and responses in flight.
At each max node, the agent can make an additional request ($\request(r)$, where $r \in \{1, \ldots, n\}$), wait for an in-flight request to return ($\wait$)\footnote{Note that we have restricted our agent to not be able to wait arbitrary amounts of time, and rather just wait till the next possible action.}
or turn in its the best guess ($\turnin$).
When an $\request(r)$ action is made, the environment samples a response $z_r$ under the current model, $z_r \sim \p(\cdot \given \bx, z_1, \dots z_n)$, and delay $\tau_r$, but the agent does not get to observe these, yet.
When a $\wait$ action is made, the environment returns the nearest response, $r$, and advances time by $\tau_r$\expand.
When the agent takes the $\turnin$ action, it receives a reward equal to the expected loss under the current state:
\begin{align*}
  \hat\sL &= \E_{\by \sim \p(\cdot \given \bx, \bz)}[\ell(\by, \byt) + C(R, \tau)],
\end{align*}
where $\bz = z_1, \dots z_m$ are the responses received when turning in, $R$ is the set of all requests that were made, and $\tau$ is the current time.

The expected utility of a node, $p$, can be computed recursively:
\begin{align*}
  U(p) &= 
  \begin{cases}
    \max\{ U(c) | c \in \textrm{children}(p) \} & \textrm{if $p$ is a max node} \\
    \sum_{c \in \textrm{children}(p)} p(c) * U(c) & \textrm{if $p$ is a expectation node}.
  \end{cases}
\end{align*}
The expectimax policy for behavior is to simply choose the optimal action at the root node: $\pi^* = \argmax_{c \in \textrm{children}(\textrm{root})} U(c)$.

\paragraph{Behavior}
Consider the example in \figureref{game-tree}.
\todo{make less abstract}


Let us study the behavior for a single request, we can evaluate the expected cost for each possible label to measure, as well as not labeling anything.

\paragraph{Efficient approximations}

However, because of time, each expectation node has an infinite number of children.
We resolve this problem by estimating value in expectation, using features, inspiration Dyna-2. \todo{more details!}

Only one action per time step.
Model time as discrete. Introduce features for states.
Stress that this is really novel in the active learning community.

\pl{Instead of writing down equations,
cast this as a game tree from the beginning with actions, possible feedback, etc.
Draw a nice example.
Then expected utility, expectimax policy, etc. should be conceptually obvious.
}

% -- EVERYTHING BELOW THIS IS DEAD TO ME. --
% Let $\ell(\by, \byt)$ be the loss incurred when if $\by$ is labeled $\byt$.
% When presented with an example $\bx$ to label, our system estimates a loss of $\E_{p(\by \given \bx)}[\ell(\by, \byt)]$, where $\byt = \argmax_{\by} \p(\by \given \bx)$.
% If we performed the measurement operator $\sigma$ and received a measurement $\tau$,
% then our expected loss would be $\E_{p(\by \given \bx, \tau)}[\ell(\by, \byt(\tau))]$, where $\byt(\tau) = \argmax_{\by} p(\by \given \bx, \tau)$.
% Intuitively, if we had perfect feedback, observing $\tau$ would provide use more information, reducing our risk.
% However, taking measurements has an associated cost, $C(\sigma)$, a function of time and money that the designer must choose.
% There is also a possibility that the measurement does not return a value (because of a timeout).
% 
% Let the CDF be $F_\sigma(t)$.
% We model the utility of a particular measurement operation, given a time window $t_0$ to be:
% \begin{align*}
% U(\sigma)
% &= F_\sigma(t_0) 
%   \E_{p(\tau \given x, \sigma)} \left[\E_{p(\by \given \bx, \tau)}[\ell(\by, \byt(\tau))] \right]
%   + (1 - F_\sigma(t_0)) 
%     \left[\E_{p(\by \given \bx)}[\ell(\by, \byt)] \right]
%   + C(\sigma).
% \end{align*}
% \pl{too abstract!  I know the measurements paper was a bit abstract...do as I say, not as I did}
% 
% Without loss of generality, assume that the null measurement is free: $C(\sigma_0) = 0$.
% Intuitively, this ensures that we will only ever choose to measure something if the expected reduction in risk is more than the cost of executing the measurement.
% 
% Let the label $\by$ have $n$ components: $\by = (y_1, ..., y_n)$.
% Further, let us assume that the loss function $\ell$ decomposes over labels: $\ell(\by, \byt) = \sum_{i=1}^n \ell(y_i, \yt_i)$. 
% Under this assumption, the expected utility of a single measurement operator $\sigma$ can be efficiently computed with $2L$ inference calls\footnote{The marginal inference query in lines 6 and 7 of \algorithmref{expected-utility} can be shared.} to the model using \algorithmref{expected-utility}.
% 
% The measurement operator to take is simply $\sigma^* = \argmin_{\sigma \in \Sigma} U(\sigma)$.
% 
% \begin{algorithm}
% \renewcommand{\algorithmicrequire}{\textbf{Input:}}
% \renewcommand{\algorithmicensure}{\textbf{Output:}}
%   \caption{Computing expected utility $U(\sigma)$}
%   \label{algo:expected-utility}
%   \begin{algorithmic}[1]
%     \REQUIRE Measurement operator $\sigma$, input $\bx$, models $p_\theta(\by \given \bx)$ and $p_\theta(\by \given \bx, \tau, \sigma)$, $F_\sigma$ and $t_0$.
%     \ENSURE Expected utility $U(\sigma)$
%     \STATE Let $y_\sigma$ be label(s) measured by operator $\sigma$.
%     \STATE Compute $p_\theta(y_\sigma \given \bx)$ using marginal inference.
%     \STATE Set $p_\theta(\tau \given \bx) \gets p_\theta(\tau \given y_\sigma, \bx) p_\theta(y_\sigma \given \bx)$.
%     \STATE Initialize $u \gets (0, \dots, 0)$.
%     \FORALL{$i \in [L]$}
%     \STATE Compute $\byt = \argmax_{\by} p_\theta(\by \given \bx, \tau = i, \sigma)$ using marginal inference.
%     \STATE Compute $p(y_j) = p_\theta(\by_j \given \bx, \tau = i, \sigma)$ for $j \in [n]$ using marginal inference.
%     \STATE Update $u[i] \gets \p(\tau = i \given \bx) \E_{p(y_j)}[\ell(y_j, \yt_j)]$.
%     \ENDFOR
%     \STATE Return the expected utility: $\frac{\sum_{i=1}^L u[i] p_\theta(\tau = i \given x)}{\sum_{i=1}^L p_\theta(\tau = i \given x)}$
%   \end{algorithmic}
% \end{algorithm}
% 
% From a practical perspective, we need to execute multiple queries. We consider this in \sectionref{async}.
% 
% 
% 
% For the system to be real-time, we need to dispatch multiple measurement queries at the same time.
% Let $\sigma_1, \sigma_2, \dots, \sigma_n$ be the set of queries we can dispatch.
% 
% \begin{note}[Baseline: Next best policy]
% \noteb{(arun): We should probably move this to experiments as a baseline or ignore all together.}
% In this scheme we do not reason about the future and choose subsequent measurement operators by going down the ordered list of measurement utilities.
% This approach, while simple, does not allow us to query the same node multiple times, which is often optimal if there is high uncertainty on a single important node.
% \end{note}
% 
% We need to reason about the possible responses that might be returned.
% For the sake of simplicity, we will choose (sequentially) the best set of $n$ measurements to make at the very beginning of our time window, not taking into account responses.
% 
% \algorithmref{expected-utility} can be trivially updated by incorporating previous measurement operators, say $\tau_1, \dots, \tau_{n-1}$.
% Naively, this would require us to enumerate over $L^d$ possible values of $\tau_1, \dots, \tau_n$ in line 5 of \algorithmref{expected-utility}.
% Instead, we propose using a particle filter to estimate utilities (\algorithmref{filtered-utility}).
% 
% \begin{algorithm}
% \renewcommand{\algorithmicrequire}{\textbf{Input:}}
% \renewcommand{\algorithmicensure}{\textbf{Output:}}
% \caption{Computing expected utility $U(\sigma_n \given \sigma_{1:n-1})$ with a particle filter}
%   \label{algo:expected-utility}
%   \begin{algorithmic}[1]
%     \REQUIRE Measurement operators $\sigma_1, \dots, \sigma_n$, input $\bx$, models $p_\theta(\by \given \bx), \dots, p_\theta(\by \given \bx, \tau_{1:n}, \sigma_{1:n})$
%     \ENSURE Expected utility $U(\sigma_n \given \sigma_{1:n-1})$
%     \STATE Let $y_{\sigma_n}$ be label(s) measured by operator $\sigma_n$.
%     \STATE Initialize $u \gets (0, \dots, 0)$.
%     \FORALL{particles $t \in [T]$}
%       \FORALL{$i \in [n-1]$}
%       \STATE Sample $\tau_i\oft{t} \sim \p(\by \given \bx, \tau_{1:i-1}\oft{t}, \sigma_{1:i})$
%       \ENDFOR
%       \STATE Set $\pi(\tau_n) \gets \p(\tau_n \given \bx, \tau_{1:n-1}\oft{t}, \sigma_{1:n})$.
%       \STATE Initialize $u\oft{t} \gets (0, \dots, 0)$.
%       \FORALL{$\tau_n \in [L]$}
%       \STATE Compute $\byt = \argmax_{\by} p_\theta(\by \given \bx, \tau_{1:n}, \sigma_{1:n})$ using MAP inference.
%       \STATE Compute $p(y_j) = p_\theta(\by_j \given \bx, \tau_{1:n}, \sigma_{1:n})$ for $j \in [n]$ using marginal inference.
%       \STATE Update $u\oft{t}[\tau_n] \gets \E_{p(y_j)}[\ell(y_j, \yt_j)]$.
%       \ENDFOR
%       \STATE Update the expected utility: $u \gets u + \frac{1}{T} \frac{\sum_{i=1}^L u[i] \pi(i)}{\sum_{i=1}^L \pi(i)}$.
%     \ENDFOR
%     \STATE Return the expected utility: $u$.
%   \end{algorithmic}
% \end{algorithm}
% 
% For each measurement, we compute the operator maximizing the expected utility $\sigma_n^* = \argmax U(\sigma_n \given \sigma_{1:n-1})$ until we reach a $\sigma_n^* = \sigma_0$.
% \todo{(arun): BAD NOTATION! The subscripts refer to members of $\Sigma$, but also the sequence of measurements.}
% 
% \pl{I guess the particle filter is out of date;
%   in any case, I think we give one algorithm
% that works on the game tree, and say what the computational complexity of the different operations}
% 
% \subsection{Modeling time}
% \label{sec:time}
% 
% When asking for multiple requests, we must decide between sending a request for information right now versus when we receive the measurement.
% In the latter case, we have more information and can make a better informed decision.
% Provided unlimited time, the latter is always optimal, but realistically, we have a finite time window in which to make decisions.
% 
% We model this.
% 
% \paragraph{Preventing overconfidence!}
% Partial monitoring tells us to just sample randomly with some epsilon rate.
% 
% Alekh~\cite{agarwal2013selective} tells us to sample a random dataset occasionally and then see if it's model is better than the actively sampled one. Sounds a bit stupid to me.
% 

\section{Modeling human error}
\label{sec:human-error}

\pl{should we have this entire section up front;
it seems like you have to talk about human error in the original framework}

Accurately identifying and hence modeling labeling noise is important if we would like to maximize the information we can get from labelers.


\todo{(arun): rewrite}

Our learner will be allowed to query humans for additional certainty about individual random variables (nodes) $X_i$.
 Humans will exist in a pool $\sP$.
 An individual human $o_i \in \sP$ (for ``oracle'', using the loosest possible definition of oracle) has a model for expected behavior.
Our learner will be allowed to query humans for additional certainty about individual random variables (nodes) $X_i$.
 Humans will exist in a pool $\sP$.
 An individual human $o_i \in \sP$ (for ``oracle'', using the loosest possible definition of oracle) has a model for expected behavior. Specifically, we model the $o_i$ expected delay to respond to a question, and the error function (i.e. response given the true state of the world $h_{\text{true}}$).

For this work we use a simplified model of error, shared uniformly across crowd workers.
 Previous work \cite{yan2011active} \cite{donmez2008proactive} \cite{golovin2010near} has shown that in an offline setting treating oracles uniformly leads to a loss, but in practice our pool is churning so quickly that we don't have time to learn accurate distinctions between workers.
 We leave a solution to this problem to future work.

When asked about variable $X_j$, human error is modeled as correct (returns $h_{\text{true}}(X_j)$) with probability $1-\epsilon$, and chosen uniformly at random from $D_j$ with probability $\epsilon$.
 We use the notation $Q(o_i, X_j)$ to denote the response from asking human $i$ about random variable $j$.

\begin{equation}
    Q(o_i, X_j) \sim
    \begin{cases}
       \text{uniformly drawn from } D_j = \{1 \ldots K_j\}, & \text{with probability}\ \epsilon \\
      h_{\text{true}}(X_j), & \text{otherwise}
    \end{cases}
 \end{equation}
 
This answer arrives after a delay.
 We model the amount of time the worker takes to answer with a gaussian, parameterized by $\mu$ and $\sigma$.
 We use the notation $\sD(o_i, X_j)$ to denote the delay in response when asking human $i$ about random variable $j$.

\[\sD(o_i, X_j) \sim \sN(\mu, \sigma^2)\]

This is an imperfect model, since it assigns some mass to a negative response time, which is impossible, but given a large $\mu$ and relatively small $\sigma$ the mass assigned to $\sD(o_i, X_j) < 0$ is negligible, and it otherwise accurately reflects human response times.

%\section{Partial Monitoring Game}
\label{sec:partial}

To cast this problem as a Partial Monitoring Game, we must first make sure the outcome space $\sS$ is finite, which requires that we discretize time into epochs of length $t_{\text{epoch}}$, and impose a time limit on the game, $t_{\text{max}}$, resulting in a finite game space $\sS_{\text{discrete}}$.

\todo{details on how Partial Monitoring Games work}

We can define the entries of our loss matrix $L$ with respect to our loss function $\sL(h_{\text{final}}, h, m, t)$.
 The language $\Sigma$ used to populate our observability matrix $H$ is defined as the state space of the game $\sS_{\text{discrete}}$.

\todo{do math... borrow bounds from Bartok et. al, show Pareto optimal policy}

Our goal is to pick a policy $\pi^*$ to minimize the loss in the terminal state of the algorithm $s_{\text{final}}$.
 Since our loss term depends on the true state of the world $\theta^* = (h_{\text{true}}, t, m)$, of which only $t$ and $m$ are ever known, we can't do this directly.
 However, we have a distribution over our beliefs $p(h_{\text{true}})$ at every step (and consequently the state of the true state of the world $\theta^*$, since $t$ and $m$ are known), so we instead minimize \textit{risk}, which is defined with respect to a particular state of the world $\tilde{\theta}$ as
\[E_{\Theta \sim p(\theta)}[Loss(\Theta || \tilde{\theta})] = \int_{\theta} Loss(\theta || \tilde{\theta})p(\theta)d\theta\]

We can formulate our goal in concrete terms as follows:
\begin{equation*}
\begin{aligned}
& \underset{\pi}{\text{minimize}}
& & E_{\pi}[E_{h \sim p(h)}[\sL(h_{\text{final}}, h, m, t)]] \\
%& \text{subject to}
%& & a \in \sA
\end{aligned}
\end{equation*}

This reduces to the Optimal Decision Tree problem, which is NP-hard.


\section{Related Work}
\label{sec:related}

\todo{(arun): This section is currently a whole page long! We need to cut somewhere}

Our work brings together ideas from the active classification and learning literature as well as literature on collective intelligence systems from the human computer interaction community.
Our focus has been to address the practically relevant questions that arise at the watershed of these two fields: how to optimally classify instances in {\em real-time} allowing {\em queries} to {\em noisy oracles}. 
In this section, we review prior work and situate our own work within the literature.

\paragraph{Active learning and active classification}
In the traditional active learning setup, we are provided a large pool of unlabeled instances from which we must chose a subset to label. The goal is to pick a subset that minimizes the risk of the classifier obtained by training on that subset. 
Active learning has also been studied in the so-called streaming setting~\cite{agarwal2013selective,cheng2013feedback,chu2011unbiased,helmbold1997some,vzliobaite2011active} where the pool of examples is divided into smaller chunks; the most useful instances within each chunk are chosen to be labeled.
In either case, such a pool of examples is not available in our setting and we must make querying decisions sequentially on each input, i.e.\ at test time.
Our decision criteria is no longer which instance is optimal to label, but whether (partially) labeling an instance is worth the cost or not.

Active classification~\cite{greiner2002learning} also makes decisions at test time, but tries to find the {\em most informative feature} to measure.
One could view our measurements as high-informative features, however our work differs in two respects: we never get to actually observe the true labels and our system is evaluated on regret as opposed to classifier risk.
Greiner et al.~\cite{greiner2002learning} study the theoretical properties of active classification with the PAC framework.
Chai et al.~\cite{chai2004test} and Esmeir et al.~\cite{esmeir2007anytime} propose algorithms for active classification in the context of naive Bayes and decision trees respectively. Both algorithms rely on having a fully labeled dataset which is used to learn when certain features should be queried.

Despite the differences highlighted above, literature on active learning and active classification is very relevant because of the application of Bayesian decision theory to graphical models. 
Settles et al.~\cite{settles2008analysis} compare different utility choices when querying for complete labels for a CRF sequence model.
There is little existing work on querying a subset of the labels within a single structured output problem.
Angeli et al.~\cite{angeli2014combining} identify instances to label within a cluster of examples in a distantly supervised setting. While this choice was a subset of the labels in the graphical model, interactions between other labels in cluster were not considered.
Liang et al.~\cite{liang09measurements} introduced the measurement framework and studied the problem of active selection of measurements. However, the measurements considered were aggregated across the dataset (e.g.\ the expected proportion of a label), rather than label measurements within an instance.

\pl{well, within an example in some sense is a special case and the easy part, so you shouldn't use this as the contrast}

Finally, existing work~\cite{donmez2008proactive,golovin2010near,yan2011active,vijayanarasimhan2014large} has considered choosing measurements from multiple noisy oracles with heterogeneous costs.
We refer the reader to \cite{settles2010active} for a survey of active learning and its variants.


% This is more of a UI sort of thing, not really useful to us.
%~\cite{roth2006active} and~\cite{culotta2005reducing}, where humans perform top-K selection over model predictions.
%The systems fall back by stages to traditional no-assistance annotation if the top-K doesn't contain the any correct information.

% Ignoring Tong et. al because it's complex and doesn't quite handle this structured thing. It's about sampling for variables conditioned on somethin.
%jit in the context of fully Bayesian networks where the oracle can draw samples conditioned on certain ``controllable'' values - introduce the notion of expected posterior risk - something we also use.


% While this approach is effective when possible, it relies on the model to consistently produce the correct answer in a top-K for some very small $K$, so for large output spaces it breaks down.

%\paragraph{Noisy oracles}
%
%There's been a line of work on Active Learning in the context of multiple noisy, expensive oracles (\cite{donmez2008proactive,golovin2010near,yan2011active,vijayanarasimhan2014large}).
%This work tries to relax the traditional assumptions in active learning that the oracle is infallible and has no economic cost.
%Some of this work is directly motivated by applications to crowd-sourcing platforms that we investigate.
%WHY DIFFERENT?

% Bayesian priors we use not.
%Finally Bayesian active learning (\cite{golovin2010near},~\cite{tong2000active}) allows us to incorporate a Bayesian prior over our data, and we'll use this as a foundation for our approach to solving the asynchronous behavior problem.

\paragraph{Collective intelligence systems}

Using crowd workers to assist labeling tasks is an area of active research within the HCI community.
\textit{Flock}~\cite{chengflock} first crowdsources the identification of features for an image classification task, and then asks the crowd to annotate examples.
Our work seeks to make the second step more cost-effective by only querying the crowd when needed.
In another line of work, \textit{TurKontrol}~\cite{peng2010decision} models individual crowd worker reliability to optimize the number of human votes needed to achieve confident consensus using a POMDP.
We also model the reliability of workers though using an unsupervised model, similar to \findcite{crowd em}.
Finally, recent work has studied how to support real-time behavior with crowd workers~\cite{bernstein2011crowds,lasecki2013real} by hiring workers ``on retainer''.
We use the same retainer model to maintain a pool of real-time crowd workers with low response times.

% Using crowds to power decision making is not a new idea. Systems in this space that support real-time behavior include \textit{Adrenaline}~\cite{bernstein2011crowds} and \textit{Legion AR}~\cite{lasecki2013real}, which both use a system where crowds are recruited ``on-retainer'' in order to be available at a moments notice.
% We use the same retainer model to maintain a pool of real-time workers with low response times.
%Using artificial-intelligence-crowd hybrids for time-insensitive workflows has also been previously explored.
%It makes no effort to train a model to augment or take over from the workers, so costs remain constant over time.
%Empirical studies have shown that this is an effective method for managing complex workflows \cite{peng2011artificial}.
%Our work follows \cite{peng2010decision} in that we apply Decision Theory to the problem of when to query the crowd, but we train a model to take over for the workers over time, and we handle the additional real time response constraint.

\paragraph{Partial monitoring games}
\noteb{This bullet point is actually to motivate that our problem is theoretically feasible.}
Our evaluation metric is unique in the active learning space in that consider a loss we do not observe because we never receive true labels.
By treating the measurements as partial feedback, our work can be theoretically modeled as a partial monitoring game\cite{cesabianchi06regret} and, in particular, an instance of the label-efficient learning problem\cite{cesabianchi05minimizing}.
Cesa-Bianchi et al.~\cite{cesabianchi06regret} show that in the online setting, the regret of a partial monitoring game is lower bounded by $O(T^{2/3})$, where $T$ is the number of examples seen. They also provide an algorithm that meets this bound: use the current model to pick the best label and query for complete labels at random with a small probability to update your model.
These guarantees provide theoretical foundation for our work.

\section{Experiments}
\label{sec:experiments}

% === Other stuff.

In this section, we empirically evaluate our approach on three tasks. 
While the on-the-job setting we propose is targeted at scenarios where there is no data to begin with, we use existing labeled datasets (\tableref{dataset}) to empirically evaluate the performance of our system relative to baselines.

\paragraph{Baselines.}
We evaluated the following four methods on each dataset:
\begin{enumerate}
  \item {\bf Human n-query}: The majority vote of $n$ human crowd workers was used as a prediction.
  \item {\bf Online learning}:
    Uses a classifier that trains on the gold output for all examples seen so far and then returns the MLE as a prediction.
    This is the best possible offline system: it sees perfect information about all the data seen so far, but can not query the crowd while making a prediction.
  \item {\bf Threshold baseline}: Uses the following heuristic:
    For each label, $y_i$, the agent asks for $m$ queries such that $(1 - \p(y_i\given \bx))\times 0.3^m \ge 0.98$. % \footnote{We found $0.88$ to give the best results on our datasets.}
    Instead of computing the expected marginals over the responses to queries in flight, the agent simply counts the in flight requests for a given variable, and reduces the uncertainty on that variable by a factor of $0.3$. The system continues launching requests until the threshold (adjusted by number of queries in flight) is crossed. Predictions are made using MLE on the model given responses.
    The agent does not reason about time and makes all its queries at the very beginning.
  \item {\bf LENSE:} Our full system as described in \sectionref{model}.
\end{enumerate}

% === Figures
\begin{table}[t]
  \begin{tabular}{l p{0.35\textwidth} p{0.35\textwidth} r r r}
    {\bf Dataset (Examples)} & {\bf Task and notes} & {\bf Features} \\ \hline
  {\bf NER (657)}     & 
    We evaluate on the CoNLL-2003 NER task\tablefootnote{\href{http://www.cnts.ua.ac.be/conll2003/ner/}{http://www.cnts.ua.ac.be/conll2003/ner/}}, a sequence labeling problem over English sentences. 
    We only consider the four tags corresponding to persons, locations, organizations or none\tablefootnote{
    The original also includes a fifth tag for miscellaneous, however the definition for miscellaneos is complex, making it very difficult for non-expert crowd workers to provide accurate labels.}.
    &
    We used standard features~\cite{finkel2005incorporating}: the current word, current lemma, previous and next lemmas, lemmas in a window of size three to the left and right, word shape and word prefix and suffixes, as well as word embeddings. \\
  {\bf Sentiment (1800)} & 
    We evaluate on a subset of the IMDB sentiment dataset \cite{maas2011learning} that consists of 2000 polar movie reviews; the goal is binary classification of documents into classes $\textsc{pos}$ and $\textsc{neg}$. 
    &
    We used two feature sets, the first (\textsc{unigrams}) containing only word unigrams, and the second (\textsc{rnn}) that also contains sentence vector embeddings from~\cite{socher2013recursive}.
    \\
  {\bf Face (1784)} & 
  We evaluate on a celebrity face classification task \cite{attribute_classifiers}. Each image must be labeled as one of the following four choices: Andersen Cooper, Daniel Craig, Scarlet Johansson or Miley Cyrus.
    &
    We used the last layer of a 11-layer AlexNet~\cite{krizhevsky2012imagenet} trained on ImageNet as input feature embeddings, though we leave back-propagating into the net to future work.
\end{tabular}
  \caption{Datasets used in this paper and number of examples we evaluate on.}
\label{tbl:dataset}
\end{table}


% To initialize parameters for the model-based methods, we used a burn-in period of 40 examples during which everything was labeled. We used those responses to train initial parameters for the prediction model $\theta$, response model $\beta$ and delay model $\Gamma$.
% We do not update parameters for the delay and response models online.

\paragraph{Implementation and crowdsourcing setup.}
We implemented the retainer model of~\cite{bernstein2011crowds} on Amazon Mechanical Turk to create a ``pool'' of crowd workers that could respond to queries in real-time.
The workers were given a short tutorial on each task before joining the pool to minimize systematic errors caused by misunderstanding the task.
We paid workers \$1.00 to join the retainer pool and an additional \$0.01 per query (for NER, since response times were much faster, we paid \$0.005 per query).
Worker response times were generally in the range of 0.5-2 seconds for NER, 10-15 seconds for Sentiment, and 1-4 seconds for Faces.

When running experiments, we found that the results varied based on the current worker quality. %, fluctuating on the NER task between 87 and 96 F1, depending on workers.
To control for variance in worker quality across our evaluations of the different methods, we collected 5 worker responses and their delays on each label ahead of time\footnote{These datasets are available in the code repository for this paper}.
During simulation we sample the worker responses and delays without replacement from this frozen pool of worker responses.

\begin{table}[t]
%% NER 
\begin{tabular}{l r r r r r r | r r r r r}
  \multicolumn{7}{c|}{Named Entity Recognition} & 
      \multicolumn{3}{c}{Face Identification} \\
      \textbf{System} & \textbf{Delay/tok} & \textbf{Qus./tok} & \textbf{PER F$_1$} & \textbf{LOC F$_1$} & \textbf{ORG F$_1$} & \textbf{F$_1$}
          & \textbf{Latency} & \textbf{Qus./ex} & \textbf{Acc.} 
    \\ \hline
    1-vote & 467 ms & 1.0 & 90.2 & 78.8 & 71.5 & 80.2
      & %1-vote & 
      1216 ms & 1.0 & 93.6 \\ %\hline
    3-vote & 750 ms & 3.0 & 93.6 & 85.1 & 74.5 & 85.4
        & %3-vote &
        1782 ms & 3.0 & 99.1 \\ %\hline
    5-vote & 1350 ms & 5.0 & \textbf{95.5} & 87.7 & 78.7 & 87.3 
        & 
        2103 ms & 5.0 & 99.8 \\ \hline
    Online & n/a & n/a & 56.9 & 74.6 & 51.4 & 60.9
        % Online 
        & n/a & n/a & 79.9 \\    %\hline
    Threshold & 414 ms & 0.61 & 95.2 & \textbf{89.8} & 79.8 & 88.3
        % threshold 
        & 1680 ms & 2.66 & 93.5 \\ %\hline
    \textbf{LENSE} & \textbf{267 ms} & \textbf{0.45} & 95.2 & 89.7 & \textbf{81.7} & \textbf{88.8} 
    %\textbf{LENSE} 
    & 1590 ms & 2.37 & 99.2 \\   %\hline
\end{tabular}
\caption{Results on NER and Face tasks comparing latencies, queries per token and performance metrics (Precision, Recall and \fone{} for NER and accuracy for Face).}
\label{tbl:results}
\end{table}

\begin{table}[t]
    \begin{minipage}[t]{.49\textwidth}
  \includegraphics[width=\textwidth]{figures/sentiment_cost_per_token_vs_time/cost_per_token_vs_time.pdf}
  \captionof{figure}{Queries per example for LENSE on Sentiment. With simple \textsc{unigram} features, the model quickly learns it does not have the capacity to satisfy our desired confidence values and must query the crowd. With more complex \textsc{rnn} features, the model learns to be more confident and queries the crowd less over time.}
        \label{fig:sentiment-tradeoff}
    \end{minipage}
    \qquad
    \begin{minipage}[t]{.49\textwidth }
        \begin{tabular}[b]{l  r  r  r  r}
    %\hline
    \textbf{System} & \textbf{Latency} & \textbf{Qu./ex} & \textbf{Acc.} \\ \hline
    1-vote & 6.6 s & 1.00 & 89.2 \\ %\hline
    3-vote & 10.9 s & 3.00 & 95.8 \\ %\hline
    5-vote & 13.5 s & 5.00 & 98.7 \\ %\hline
    \multicolumn{5}{c}{\textsc{bigrams}} \\ \hline
%Bigrams: \\
    Offline & n/a & n/a & TODO \\ %\hline
    \textbf{Threshold} & TODO s & TODO & TODO \\ %\hline
    LENSE & 18.1 s & 3.48 & 98.6 \\ %\hline
    \multicolumn{5}{c}{\textsc{rnn}} \\ \hline
    Offline & n/a & n/a & TODO \\ %\hline
    \textbf{Threshold} & 11.0 s & 2.85 & 96.0 \\ %\hline
    \textbf{LENSE} & 11.0 s & 3.19 & 98.6 \\% \hline
\end{tabular}

        \caption{Results on the Sentiment task comparing latency, queries per example and accuracy.}
        \label{tbl:sentiment-results}
        \vfill
    \end{minipage}%
\end{table}

% DETAILS
% Anecdotally, we also report a range of results on 5 complete runs of our system using {\em real live crowd workers}, recruited at test time, over the first 150 sentences of our dataset. The results exhibit high variance based on worker quality.
%
%\begin{center}
%\begin{tabular}{ | r | r | r | r | r | }
%    \hline
%    Time/token & Requests/token & Precision & Recall & F1 \\ \hline
%    1444 ms - 3426 ms & 0.54 - 0.66 & 92.9 - 96.91 & 82.50 - 94.01 & 87.4 - 95.43 \\ \hline
%\end{tabular}
%\end{center}
%
%Filtering workers while running, or inferring separate error models for each worker, would clearly deliver substantial gains in reliability over a system that assumes uniform quality. We leave this to future work.

\paragraph{Summary of results.}
\tableref{results} and \tableref{sentiment-results} summarize the performance of the methods on the three tasks.
On all three datasets, we found that on-the-job learning outperforms machine and human-only comparisons on both quality and cost. 
On NER, we achieve an \fone{} of $88.4\%$ at more than an order of magnitude reduction on the cost of achieving comporable quality result using the 5-vote approach. On Sentiment and Faces, we reduce costs for a comparable accuracy by a factor of 2-3.
For the latter two tasks, both on-the-job learning methods perform less well than in NER. We suspect this is due to the presence of a dominant class (``none'') in NER that the model can very quickly learn to expend almost no effort on, and so profit.

\begin{figure}[t]
  \centering
  \begin{subfigure}[b]{0.49\textwidth}
  \includegraphics[width=\textwidth]{figures/ner_2_class/f1_plot/f1_vs_time.pdf}
\end{subfigure}
  \begin{subfigure}[b]{0.49\textwidth}
  \includegraphics[width=\textwidth]{figures/ner_2_class/cost_plot/cost_vs_time.pdf}
  \end{subfigure}
  \caption{Comparing \fone{} and queries per token on the NER task over time. The left graph compares LENSE to online learning (which cannot query humans at test time). This highlights that LENSE maintains high \fone{} scores even with very small training set sizes, by falling back the crowd when it is unsure. The right graph compares query rate over time to 1-vote. This clearly shows that as the model learns, it needs to query the crowd less.}
\label{fig:ner-f1}
\end{figure}

%\paragraph{Does the model respect accuracy preferences?} 
\figureref{ner-f1} tracks the performance and cost of LENSE over time on the NER task.
LENSE is not only able  to consistently outperform other baselines, but the cost of the system steadily reduces over time.
On the NER task, we find that LENSE is able to trade off time to produce more accurate results than the threshold baseline with fewer queries by waiting for responses before making another query.

% == DEV F1
%We also plot \fone{} on a held-out set: while we receive noisy and sparse supervision LENSE is able to train a classifier that generalizes well over time. 
%Compared to the offline baseline, which sees perfect labels on all examples, LENSE observed labels on only \ac{KEENON: fix this number: 600} words. For a comparable number of tokens, the offline baseline has an \fone of \ac{KEENON: fix: 63\%}.

% === Compare with threshold
%\paragraph{Why is the threshold system so much better on sentiment and faces?}
%LENSE exploits structure in the model when querying, and so outperforms the threshold in the presence of structure (see NER task results).
%However, when neither structure or time pressure is present, LENSE's only signal is the entropy of the prediction, and so the threshold is able to perform just as well, if not slightly better when the threshold is optimally set.

%We note that although our model is receiving supervisition signal at roughly 1/15 the rate of a fully observed online learning scenario\footnote{this calculation assumes ``fully observed'' is approximated by 3 human labels per example}, we track roughly parallel to the fully supervised line.

%\paragraph{In the limit, would LENSE still need crowd supervision?}
While on-the-job learning allows us to deploy quickly and ensure good results, we would like to eventually operate without crowd supervision.
%In order to do so, we must ensure that our model has the capacity to keep learning from the crowd.
\figureref{sentiment-tradeoff}, we show the number of queries per example on Sentiment with two different features sets, \textsc{unigrams} and \textsc{rnn} (as described in \tableref{dataset}).
With simpler features (\textsc{unigrams}),
the model saturates early and we will continue to need to query to the crowd to achieve our accuracy target (as specified by the loss function).
On the other hand, using richer features (\textsc{rnn}) the model is able to learn from the crowd and the amount of supervision needed reduces over time.
%This suggests that given a sufficiently rich model, costs can be brought to zero in the limit.
Note that even when the model capacity is limited, LENSE is able to guarantee a consistent, high level of performance.

% -- CUT
%\paragraph{When does LENSE query?}
%During the initial stages of learning for the NER task, LENSE made multiple queries on all tokens.
%After a few examples, though, LENSE focused its queries on unseen entity tokens.
%Consider the following example taken from our run logs: \textit{``U.S.\ says still committed to Cuba migration pacts''}.
%%For example, after seeing \ac{600} examples, 
%The model has already seen {\it U.S.\/} and predicts that it is a \scloc{} with 97\% probability.
%On the other hand, having never seen the token {\it Cuba\/} before, the model starts with a belief that {\it Cuba\/} is 59\% \scper, 37\% \scloc{} and 3\% \scnone.
%It immediately fires off two queries about {\it Cuba\/} and waits 4 seconds for both responses to return \scloc{}.
%The model now believes that {\it Cuba\/} is \scloc{} with 95\% probability and returns its prediction.
%This demonstrates that using the model allows LENSE to focus supervision to where it is needed.


% \paragraph{Are we learning a good model?}
% We receive very noisy and sparse supervision.
% Despite this, to reduce the costs in the future, our system must learn to generalize well.
% We refer to \figureref{ner-dev-f1}.
% We note that although our model is receiving supervisition signal at roughly 1/15 the rate of a fully observed online learning scenario\footnote{this calculation assumes ``fully observed'' is approximated by 3 human labels per example}, we track roughly parallel to the fully supervised line.

% Held out figure for space concerns.
%\begin{figure}[t]
%  \begin{centering}
%  \includegraphics[width=1.0\textwidth]{figures/ner_2_class/machine_f1_plot/machine_f1_vs_time.pdf}
%  \end{centering}
%  \caption{A figure showing the relationship between the classifier we train, and one trained on gold labeled data. Evaluation is on a held out dev set. Note that our classifier learns more slowly because we are handing it a noisy approximation to train on, but the classifier narrows the gap as it gets more data.}
%\label{fig:ner-dev-f1}
%\end{figure}


\section{Conclusion}
\label{sec:conclusion}

We have introduced a new framework to deploy a learning system that uses crowd supervision to train a model {\em on-the-job}.
We addressed the problem of deciding what supervision to ask the crowd for by modeling the problem as a stochastic game played by the model and the crowd. 
Using Monte Carlo tree search to approximate the optimal policy, we have built a system, LENSE, that is able to trade off time and money for accuracy.
We have shown that learning on-the-job can achieve comparable accuracies to asking humans without. 

An increasing number of problems that challenge AI today involve interactive tasks that are not suitable to conventional dataset collection, for example, dialogue or question answering.
We envision on-the-job training as enabling these tasks.
Of course, there are many problems for which it is hard to ask turkers to give labels. Future work: ask for partial supervision through the measurements framework.



% I don't think we need this just yet.
%\subsubsection*{Acknowledgments}
%
% Kelvin, Volodymyr for useful discussions.
% Amy for vision experiments.
% Anonymous reviewers for their helpful feedback.

\subsubsection*{References}

\bibliographystyle{plain}
\bibliography{ref,all}

\end{document}
