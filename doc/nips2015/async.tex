\section{Choosing when and what to query}
\label{sec:async}

\pl{I'd move some of the setup before the model section,
but explain the setup more}

Our goal is to choose our querying strategy to minimize an arbitrary loss function:
\[\sL(h_{\text{true}}, h, m, t)\]
$\sL$ takes as parameters the true state of the world $h_{\text{true}}$, our guess $h$, and the money $m$ and time $t$ spent arriving at our guess.
 By tweaking $\sL$ practitioners applying our framework can move to arbitrary points on the cost-delay-accuracy tradeoff surface.
 We deliberately make no assumptions about the form of $\sL$ while developing theory, and demonstrate several choices in our experiments.
 \pl{get off of that high horse and give concrete examples!}

\pl{Instead of writing down equations,
cast this as a game tree from the beginning with actions, possible feedback, etc.
Draw a nice example.
Then expected utility, expectimax policy, etc. should be conceptually obvious.
}

Let $\ell(\by, \byt)$ be the loss incurred when if $\by$ is labeled $\byt$.
When presented with an example $\bx$ to label, our system estimates a loss of $\E_{p(\by \given \bx)}[\ell(\by, \byt)]$, where $\byt = \argmax_{\by} \p(\by \given \bx)$.
If we performed the measurement operator $\sigma$ and received a measurement $\tau$,
then our expected loss would be $\E_{p(\by \given \bx, \tau)}[\ell(\by, \byt(\tau))]$, where $\byt(\tau) = \argmax_{\by} p(\by \given \bx, \tau)$.
Intuitively, if we had perfect feedback, observing $\tau$ would provide use more information, reducing our risk.
However, taking measurements has an associated cost, $C(\sigma)$, a function of time and money that the designer must choose.
There is also a possibility that the measurement does not return a value (because of a timeout).

Let the CDF be $F_\sigma(t)$.
We model the utility of a particular measurement operation, given a time window $t_0$ to be:
\begin{align*}
U(\sigma)
&= F_\sigma(t_0) 
  \E_{p(\tau \given x, \sigma)} \left[\E_{p(\by \given \bx, \tau)}[\ell(\by, \byt(\tau))] \right]
  + (1 - F_\sigma(t_0)) 
    \left[\E_{p(\by \given \bx)}[\ell(\by, \byt)] \right]
  + C(\sigma).
\end{align*}
\pl{too abstract!  I know the measurements paper was a bit abstract...do as I say, not as I did}

Without loss of generality, assume that the null measurement is free: $C(\sigma_0) = 0$.
Intuitively, this ensures that we will only ever choose to measure something if the expected reduction in risk is more than the cost of executing the measurement.

Let the label $\by$ have $n$ components: $\by = (y_1, ..., y_n)$.
Further, let us assume that the loss function $\ell$ decomposes over labels: $\ell(\by, \byt) = \sum_{i=1}^n \ell(y_i, \yt_i)$. 
Under this assumption, the expected utility of a single measurement operator $\sigma$ can be efficiently computed with $2L$ inference calls\footnote{The marginal inference query in lines 6 and 7 of \algorithmref{expected-utility} can be shared.} to the model using \algorithmref{expected-utility}.

The measurement operator to take is simply $\sigma^* = \argmin_{\sigma \in \Sigma} U(\sigma)$.

\begin{algorithm}
\renewcommand{\algorithmicrequire}{\textbf{Input:}}
\renewcommand{\algorithmicensure}{\textbf{Output:}}
  \caption{Computing expected utility $U(\sigma)$}
  \label{algo:expected-utility}
  \begin{algorithmic}[1]
    \REQUIRE Measurement operator $\sigma$, input $\bx$, models $p_\theta(\by \given \bx)$ and $p_\theta(\by \given \bx, \tau, \sigma)$, $F_\sigma$ and $t_0$.
    \ENSURE Expected utility $U(\sigma)$
    \STATE Let $y_\sigma$ be label(s) measured by operator $\sigma$.
    \STATE Compute $p_\theta(y_\sigma \given \bx)$ using marginal inference.
    \STATE Set $p_\theta(\tau \given \bx) \gets p_\theta(\tau \given y_\sigma, \bx) p_\theta(y_\sigma \given \bx)$.
    \STATE Initialize $u \gets (0, \dots, 0)$.
    \FORALL{$i \in [L]$}
    \STATE Compute $\byt = \argmax_{\by} p_\theta(\by \given \bx, \tau = i, \sigma)$ using marginal inference.
    \STATE Compute $p(y_j) = p_\theta(\by_j \given \bx, \tau = i, \sigma)$ for $j \in [n]$ using marginal inference.
    \STATE Update $u[i] \gets \p(\tau = i \given \bx) \E_{p(y_j)}[\ell(y_j, \yt_j)]$.
    \ENDFOR
    \STATE Return the expected utility: $\frac{\sum_{i=1}^L u[i] p_\theta(\tau = i \given x)}{\sum_{i=1}^L p_\theta(\tau = i \given x)}$
  \end{algorithmic}
\end{algorithm}

From a practical perspective, we need to execute multiple queries. We consider this in \sectionref{async}.



For the system to be real-time, we need to dispatch multiple measurement queries at the same time.
Let $\sigma_1, \sigma_2, \dots, \sigma_n$ be the set of queries we can dispatch.

\begin{note}[Baseline: Next best policy]
\noteb{(arun): We should probably move this to experiments as a baseline or ignore all together.}
In this scheme we do not reason about the future and choose subsequent measurement operators by going down the ordered list of measurement utilities.
This approach, while simple, does not allow us to query the same node multiple times, which is often optimal if there is high uncertainty on a single important node.
\end{note}

We need to reason about the possible responses that might be returned.
For the sake of simplicity, we will choose (sequentially) the best set of $n$ measurements to make at the very beginning of our time window, not taking into account responses.

\algorithmref{expected-utility} can be trivially updated by incorporating previous measurement operators, say $\tau_1, \dots, \tau_{n-1}$.
Naively, this would require us to enumerate over $L^d$ possible values of $\tau_1, \dots, \tau_n$ in line 5 of \algorithmref{expected-utility}.
Instead, we propose using a particle filter to estimate utilities (\algorithmref{filtered-utility}).

\begin{algorithm}
\renewcommand{\algorithmicrequire}{\textbf{Input:}}
\renewcommand{\algorithmicensure}{\textbf{Output:}}
\caption{Computing expected utility $U(\sigma_n \given \sigma_{1:n-1})$ with a particle filter}
  \label{algo:expected-utility}
  \begin{algorithmic}[1]
    \REQUIRE Measurement operators $\sigma_1, \dots, \sigma_n$, input $\bx$, models $p_\theta(\by \given \bx), \dots, p_\theta(\by \given \bx, \tau_{1:n}, \sigma_{1:n})$
    \ENSURE Expected utility $U(\sigma_n \given \sigma_{1:n-1})$
    \STATE Let $y_{\sigma_n}$ be label(s) measured by operator $\sigma_n$.
    \STATE Initialize $u \gets (0, \dots, 0)$.
    \FORALL{particles $t \in [T]$}
      \FORALL{$i \in [n-1]$}
      \STATE Sample $\tau_i\oft{t} \sim \p(\by \given \bx, \tau_{1:i-1}\oft{t}, \sigma_{1:i})$
      \ENDFOR
      \STATE Set $\pi(\tau_n) \gets \p(\tau_n \given \bx, \tau_{1:n-1}\oft{t}, \sigma_{1:n})$.
      \STATE Initialize $u\oft{t} \gets (0, \dots, 0)$.
      \FORALL{$\tau_n \in [L]$}
      \STATE Compute $\byt = \argmax_{\by} p_\theta(\by \given \bx, \tau_{1:n}, \sigma_{1:n})$ using MAP inference.
      \STATE Compute $p(y_j) = p_\theta(\by_j \given \bx, \tau_{1:n}, \sigma_{1:n})$ for $j \in [n]$ using marginal inference.
      \STATE Update $u\oft{t}[\tau_n] \gets \E_{p(y_j)}[\ell(y_j, \yt_j)]$.
      \ENDFOR
      \STATE Update the expected utility: $u \gets u + \frac{1}{T} \frac{\sum_{i=1}^L u[i] \pi(i)}{\sum_{i=1}^L \pi(i)}$.
    \ENDFOR
    \STATE Return the expected utility: $u$.
  \end{algorithmic}
\end{algorithm}

For each measurement, we compute the operator maximizing the expected utility $\sigma_n^* = \argmax U(\sigma_n \given \sigma_{1:n-1})$ until we reach a $\sigma_n^* = \sigma_0$.
\todo{(arun): BAD NOTATION! The subscripts refer to members of $\Sigma$, but also the sequence of measurements.}

\pl{I guess the particle filter is out of date;
  in any case, I think we give one algorithm
that works on the game tree, and say what the computational complexity of the different operations}

\subsection{Modeling time}
\label{sec:time}

When asking for multiple requests, we must decide between sending a request for information right now versus when we receive the measurement.
In the latter case, we have more information and can make a better informed decision.
Provided unlimited time, the latter is always optimal, but realistically, we have a finite time window in which to make decisions.

We model this.

\paragraph{Preventing overconfidence!}
Partial monitoring tells us to just sample randomly with some epsilon rate.

Alekh~\cite{agarwal2013selective} tells us to sample a random dataset occasionally and then see if it's model is better than the actively sampled one. Sounds a bit stupid to me.

