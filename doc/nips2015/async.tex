\section{Multiple asynchronous requests}
\label{sec:async}

For the system to be real-time, we need to dispatch multiple measurement queries at the same time.
Let $\sigma_1, \sigma_2, \dots, \sigma_n$ be the set of queries we can dispatch.

\begin{note}[Baseline: Next best policy]
\noteb{(arun): We should probably move this to experiments as a baseline or ignore all together.}
In this scheme we do not reason about the future and choose subsequent measurement operators by going down the ordered list of measurement utilities.
This approach, while simple, does not allow us to query the same node multiple times, which is often optimal if there is high uncertainty on a single important node.
\end{note}

We need to reason about the possible responses that might be returned.
For the sake of simplicity, we will choose (sequentially) the best set of $n$ measurements to make at the very beginning of our time window, not taking into account responses.

\algorithmref{expected-utility} can be trivially updated by incorporating previous measurement operators, say $\tau_1, \dots, \tau_{n-1}$.
Naively, this would require us to enumerate over $L^d$ possible values of $\tau_1, \dots, \tau_n$ in line 5 of \algorithmref{expected-utility}.
Instead, we propose using a particle filter to estimate utilities (\algorithmref{filtered-utility}).

\begin{algorithm}
\renewcommand{\algorithmicrequire}{\textbf{Input:}}
\renewcommand{\algorithmicensure}{\textbf{Output:}}
\caption{Computing expected utility $U(\sigma_n \given \sigma_{1:n-1})$ with a particle filter}
  \label{algo:expected-utility}
  \begin{algorithmic}[1]
    \REQUIRE Measurement operators $\sigma_1, \dots, \sigma_n$, input $\bx$, models $p_\theta(\by \given \bx), \dots, p_\theta(\by \given \bx, \tau_{1:n}, \sigma_{1:n})$
    \ENSURE Expected utility $U(\sigma_n \given \sigma_{1:n-1})$
    \STATE Let $y_{\sigma_n}$ be label(s) measured by operator $\sigma_n$.
    \STATE Initialize $u \gets (0, \dots, 0)$.
    \FORALL{particles $t \in [T]$}
      \FORALL{$i \in [n-1]$}
      \STATE Sample $\tau_i\oft{t} \sim \p(\by \given \bx, \tau_{1:i-1}\oft{t}, \sigma_{1:i})$
      \ENDFOR
      \STATE Set $\pi(\tau_n) \gets \p(\tau_n \given \bx, \tau_{1:n-1}\oft{t}, \sigma_{1:n})$.
      \STATE Initialize $u\oft{t} \gets (0, \dots, 0)$.
      \FORALL{$\tau_n \in [L]$}
      \STATE Compute $\byt = \argmax_{\by} p_\theta(\by \given \bx, \tau_{1:n}, \sigma_{1:n})$ using MAP inference.
      \STATE Compute $p(y_j) = p_\theta(\by_j \given \bx, \tau_{1:n}, \sigma_{1:n})$ for $j \in [n]$ using marginal inference.
      \STATE Update $u\oft{t}[\tau_n] \gets \E_{p(y_j)}[\ell(y_j, \yt_j)]$.
      \ENDFOR
      \STATE Update the expected utility: $u \gets u + \frac{1}{T} \frac{\sum_{i=1}^L u[i] \pi(i)}{\sum_{i=1}^L \pi(i)}$.
    \ENDFOR
    \STATE Return the expected utility: $u$.
  \end{algorithmic}
\end{algorithm}

For each measurement, we compute the operator maximizing the expected utility $\sigma_n^* = \argmax U(\sigma_n \given \sigma_{1:n-1})$ until we reach a $\sigma_n^* = \sigma_0$.
\todo{(arun): BAD NOTATION! The subscripts refer to members of $\Sigma$, but also the sequence of measurements.}

\subsection{Modeling time}
\label{sec:time}

When asking for multiple requests, we must decide between sending a request for information right now versus when we receive the measurement.
In the latter case, we have more information and can make a better informed decision.
Provided unlimited time, the latter is always optimal, but realistically, we have a finite time window in which to make decisions.

We model this.

