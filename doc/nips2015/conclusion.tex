\section{Conclusion}
\label{sec:conclusion}

We have introduced a new framework to deploy a learning system that uses crowd supervision to train a model {\em on-the-job}.
We addressed the problem of deciding what supervision to ask the crowd for by modeling the problem as a stochastic game played by the model and the crowd. 
Using Monte Carlo tree search to approximate the optimal policy, we have built a system, LENSE, that is able to trade off time and money for accuracy.
We have shown that learning on-the-job can achieve comparable accuracies to asking humans without. 

An increasing number of problems that challenge AI today involve interactive tasks that are not suitable to conventional dataset collection, for example, dialogue or question answering.
We envision on-the-job training as enabling these tasks.
Of course, there are many problems for which it is hard to ask turkers to give labels. Future work: ask for partial supervision through the measurements framework.

