\section{Conclusion}
\label{sec:conclusion}

We have introduced a new framework that learns from (noisy) crowds \emph{on-the-job}
to maintain high accuracy, and reducing cost significantly over time.
The technical core of our approach is modeling the on-the-job setting
as a stochastic game and using ideas from game playing to approximate the optimal policy.
%We addressed the problem of deciding what supervision to ask the crowd for by
%modeling the problem as a stochastic game played by the model and the crowd. 
%Using Monte Carlo tree search to approximate the optimal policy,
We have built a system, LENSE, %, that is able to trade off time and money for accuracy,
which obtains significant cost reductions over a pure crowd approach
and significant accuracy improvements over a pure ML approach.
%We have shown that learning on-the-job can achieve comparable accuracies to
%asking humans without. 

% PL: not needed
%An increasing number of problems that challenge AI today involve interactive tasks that are not suitable to conventional dataset collection, for example, dialogue or question answering.
%We envision on-the-job training as enabling these tasks.
%Of course, there are many problems for which it is hard to ask turkers to give labels. Future work: ask for partial supervision through the measurements framework.
