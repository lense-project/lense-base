\begin{block}{Example: named entity recognition on tweets}
  \begin{center}
    \includegraphics[width=0.79\columnwidth]{soup-example.png}
  \end{center}
\end{block}
\vfill

\begin{block}{How marginals evolve after incorporating responses}
  \begin{center}
    \includegraphics[width=0.79\columnwidth]{behavior-poster}
  \end{center}
\end{block}
\vfill

\begin{block}{Approximating utility with MCTS}
  \splitcolumn{
    \begin{center}
      \includegraphics[width=0.89\columnwidth]{mcts_poster}
    \end{center}
  }{
  \begin{itemize}
    \item \textbf{Stochastic game} between system and crowd.
    \item \textbf{States} capture time, queries in flight and received responses.
    \item \textbf{Actions} are querying for a label, waiting or returning current best guess.
  \end{itemize}
  }
  \begin{itemize}
    \item The system chooses actions that \textbf{maximize utility}.
    \item Approximated by Markov Chain Tree Search (MCTS) with progressive widening, using an environment model.
  \begin{align*}
    \label{eqn:dynamics}
  %p(\by, \br, \bt \mid \bx, \bq) \eqdef \p(\by \given \bx) \prod_{i=1}^k \presp(r_i \mid \bx, \by, q_i) \ptime(t_i \mid \bx, \by, q_i, s_i).
  \underbrace{p(\by, \br, \bt \given \bx, \bq, \bs)}_{\textrm{environment model}} \eqdef \underbrace{\p(\by \given \bx)}_{\textrm{prediction}} \prod_{i=1}^k \underbrace{\presp(r_i \mid y_{q_i})}_{\textrm{annotator noise}} \underbrace{\ptime(t_i \mid s_i)}_{\textrm{latency}}.
  \end{align*}
    \item Use human-labelled examples as training data to learn the model.
  \end{itemize}

\end{block}
\vfill

%\begin{block}{Model: a stochastic game between the system and the crowd.}
%  \begin{itemize}
%    \item The game starts with the system receiving input $\bx$ and ends when the system turns in a set of labels $\by = (y_1, \ldots, y_n)$. 
%    \item During the system's turn, the system may choose a query action $q \in \{1, \ldots, n\}$ to ask the crowd to label $y_q$. The system may also choose 
%the wait action ($q = \acwait$) to wait for the crowd to respond to a pending query
%or
%the return action ($q = \acret$) to terminate the game and return its prediction given responses received thus far.
%\item
%The system can make as many queries in a row (i.e.\ simultaneously) as it wants, before deciding to wait or turn in.
%\item When the wait action is chosen, the turn switches to the crowd, which provides a response $r$ to one pending query, and advances the game clock by the time taken for the crowd to respond. The turn then immediately reverts back to the system.
%\item When the game ends (the system chooses the return action), the system evaluates a utility that depends on the accuracy of its prediction,
%the number of queries issued and the total time taken.
%\item The system should choose query and wait actions to
%maximize the utility of the prediction eventually returned.
%\item \textbf{Utility} trades off the accuracy of the MAP estimate according to the model's best guess of $\by$ incorporating all responses received by time $\tau$ and the cost of making queries: a (monetary) cost $\weightmoney$ per query made and penalty of $\weighttime$ per unit of time taken.
%\begin{align}
%  \label{eqn:utility}
%  U(\sigma) &\eqdef \text{ExpAcc}(p(\by \given \bx, \bq, \bs, \br, \bt)) - (n_\text{Q} \weightmoney + \now \weighttime), \\
%  \text{ExpAcc}(p) &= \E_{p(\by)}[\text{Accuracy}(\arg\max_{\by'} p(\by'))],
%\end{align}
%\item \textbf{Environmnet model}
%\begin{align}
%  \label{eqn:dynamics}
%%p(\by, \br, \bt \mid \bx, \bq) \eqdef \p(\by \given \bx) \prod_{i=1}^k \presp(r_i \mid \bx, \by, q_i) \ptime(t_i \mid \bx, \by, q_i, s_i).
%p(\by, \br, \bt \given \bx, \bq, \bs) \eqdef \p(\by \given \bx) \prod_{i=1}^k \presp(r_i \mid y_{q_i}) \ptime(t_i \mid s_i).
%\end{align}
%
%  \end{itemize}
%\end{block}
%
%\begin{block}{}
%  \begin{center}
%    \includegraphics[width=0.59\columnwidth]{behavior}
%    \includegraphics[width=0.39\columnwidth]{mcts_simple}
%  \end{center}
%\end{block}
%
%\begin{block}{Learning and inference}
%  \begin{itemize}
%    \item TODO insert algorithm.
%  \end{itemize}
%\end{block}

